\documentclass[letterpaper]{article}
\usepackage{aaai25}
\usepackage{times}
\usepackage{helvet}
\usepackage{courier}
\usepackage[hyphens]{url}
\usepackage{graphicx}
\usepackage{amsmath}
\usepackage{booktabs}
\usepackage{multirow}
\urlstyle{rm}
\def\UrlFont{\rm}
\usepackage{natbib}
\usepackage{caption}
\frenchspacing
\setlength{\pdfpagewidth}{8.5in}
\setlength{\pdfpageheight}{11in}

\title{Defect Detection Benchmark: A Comprehensive Benchmark for Transformer-Based Vulnerability Detection Models}

\author{
Jake Norton\\
University of Otago\\
norja159@student.otago.ac.nz
}
% \author{
% David Eyers\\
% University of Otago\\
% david.eyers@otago.ac.nz
% }
% \author{
% Veronica Liesaputra\\
% University of Otago\\
% veronica.liesaputra@otago.ac.nz
% }

\begin{document}

\maketitle

\begin{abstract}
	We present Defect Detection Benchmark, a comprehensive benchmarking framework that evaluates transformer-based models for code vulnerability detection across eight diverse datasets. Our systematic evaluation of CodeBERT, GraphCodeBERT, CodeT5, and NatGen reveals significant performance variations across datasets. Key findings include: (1) threshold optimization can improve F1 scores by up to `dummy value`, (2) dataset quality varies dramatically, with some datasets showing evidence of label leakage, (3) standardized evaluation protocols are essential for fair model comparison and (4) Dataset automation is a continually awkward problem, with datasets often being stored on google drives or other unstable platforms.(5)To my knowledge first benchmarks for the ICVUL dataset. Defect Detection Benchmark provides researchers with a unified framework for reproducible vulnerability detection research.
\end{abstract}

\section{Introduction}

Code vulnerability detection has become increasingly important as software systems grow in complexity. Recent advances in transformer-based models have shown promising results, but comparing these models fairly remains challenging due to inconsistent evaluation protocols and dataset variations.

Existing benchmarks often focus on single datasets or limited model comparisons, making it difficult to assess true model capabilities. Furthermore, many studies overlook critical factors such as threshold optimization and dataset quality, leading to potentially misleading conclusions.

Defect Detection Benchmark addresses these limitations by providing a standardized evaluation framework that systematically compares multiple transformer-based models across diverse vulnerability detection datasets while highlighting important evaluation considerations often overlooked in prior work.

\section{Defect Detection Benchmark Framework}

\subsection{Model Architecture Coverage}
Defect Detection Benchmark evaluates four representative transformer architectures:

\begin{itemize}
	\item \textbf{CodeBERT}: A RoBERTa-based model pre-trained on code and natural language, serving as our baseline due to its widespread adoption.

	\item \textbf{GraphCodeBERT}: Extends CodeBERT by incorporating code structure through data flow graphs, representing graph-enhanced approaches.

	\item \textbf{CodeT5}: A T5-based encoder-decoder model specifically designed for code understanding and generation tasks.

	\item \textbf{NatGen}: A specialized T5-based model focused on natural language descriptions of code behavior for vulnerability detection.
\end{itemize}

\subsection{Dataset Diversity}
Our benchmark includes eight vulnerability detection datasets spanning different domains and characteristics:
\begin{itemize}
	\item \textbf{DIVERSEVUL}: Real-world vulnerabilities with severe class imbalance (1:17 ratio)
	\item \textbf{Devign}: Qemu and ffmpeg functions with errors, json with difficult realworld code
	\item \textbf{ICVUL}: Intermediate-scale dataset with calibration challenges
	\item \textbf{BIGVUL}: Large-scale dataset revealing data quality issues
	\item \textbf{MVDSC}: High-performance dataset with potential synthetic characteristics
	\item \textbf{CVEFixes}: CVE-based vulnerability patches
	\item \textbf{Draper}: DARPA CGC challenge dataset
	\item \textbf{Vuldeepecker}: From SARD dataset, c/c++, similar to juliet but only buffer and resource management mix of synthetic and real?
	\item \textbf{Juliet}: NIST synthetic test suite
	\item \textbf{Reveal}: Function-level vulnerability detection
\end{itemize}

\subsection{Evaluation Protocol}
Defect Detection Benchmark implements a rigorous evaluation protocol addressing common pitfalls:

\begin{itemize}
	\item \textbf{Threshold Optimization}: We perform systematic threshold tuning to find optimal operating points, as default thresholds (0.5) are often suboptimal for imbalanced datasets.

	\item \textbf{Class Imbalance Handling}: We incorporate balanced loss weighting using pos\_weight parameters to address severe class imbalances.

	\item \textbf{Comprehensive Metrics}: Beyond accuracy, we report F1-score, precision, recall, and AUC-ROC to provide complete performance profiles.
\end{itemize}

\section{Key Findings}

\subsection{Threshold Optimization Impact}

Our evaluation reveals that threshold optimization is crucial for fair model comparison. Table~\ref{tab:results} shows performance improvements of up to 58\% in F1-score when using optimized thresholds versus default values.

\begin{table}[ht]
	\centering
	\caption{Model Performance Across Datasets (F1-Score \%)}
	\label{tab:results}
	\begin{tabular}{lccc}
		\toprule
		Dataset    & CodeBERT             & GraphCodeBERT & CodeT5 \\%& Best  \\
		\midrule
		DIVERSEVUL & \textit{In Progress} & -             & -      \\%& 25.09 \\
		ICVUL      & \textit{In Progress} & -             & -      \\%& 57.82 \\
		BIGVUL     & \textit{In Progress} & -             & -      \\%& -     \\
		MVDSC      & \textit{In Progress} & -             & -      \\%& 80.60 \\
		CVEFixes   & \textit{In Progress} & -             & -      \\%& -     \\
		Draper     & \textit{In Progress} & -             & -      \\%& -     \\
		Juliet     & \textit{In Progress} & -             & -      \\%& -     \\
		Reveal     & \textit{In Progress} & -             & -      \\%& -     \\
		\bottomrule
	\end{tabular}
\end{table}

\subsection{Dataset Quality Assessment}
Defect Detection Benchmark uncovered significant quality variations across datasets:

\textbf{DIVERSEVUL}: Represents realistic performance with moderate F1 scores reflecting real-world complexity.

\subsection{Architecture-Specific Insights}
T5-based models (CodeT5, NatGen) require different handling compared to RoBERTa-based models (CodeBERT, GraphCodeBERT), particularly in loss computation and tokenization approaches. This architectural diversity necessitates careful implementation to ensure fair comparison.

\section{Demo System}

The Defect Detection Benchmark demo system provides an interactive interface for:

\begin{itemize}
	\item \textbf{Model Comparison}: Side-by-side performance comparison across all datasets with configurable metrics and threshold settings.

	\item \textbf{Dataset Analysis}: Interactive exploration of dataset characteristics, class distributions, and quality indicators.

	\item \textbf{Threshold Optimization}: Real-time threshold tuning with immediate performance feedback across multiple metrics.

	\item \textbf{Reproducible Experiments}: Standardized training and evaluation scripts enabling consistent model comparison.

\end{itemize}

The system is designed for researchers to quickly assess model performance on new datasets and practitioners to select appropriate models for specific vulnerability detection tasks.

\section{Impact and Applications}

Defect Detection Benchmark addresses critical gaps in vulnerability detection evaluation by providing standardized benchmarks that reveal the importance of proper evaluation methodology. Our findings demonstrate that:

\begin{enumerate}
	\item Many existing comparisons may be unfair due to inadequate threshold optimization
	\item Dataset quality assessment is essential before drawing conclusions about model performance
	\item Architectural differences require careful handling for meaningful comparison
\end{enumerate}

These insights have immediate implications for both research and practical deployment of vulnerability detection systems.

\section{Conclusion}

Defect Detection Benchmark provides the first comprehensive benchmarking framework for transformer-based vulnerability detection models, revealing critical evaluation issues often overlooked in current research. Our systematic approach demonstrates the importance of proper threshold optimization and dataset quality assessment for fair model comparison.

The framework is available as an open-source tool, enabling researchers to conduct reproducible experiments and practitioners to make informed model selection decisions. Future work will expand the model coverage and integrate additional evaluation metrics based on community feedback.

\bibliographystyle{aaai}
\bibliography{references}

\end{document}
